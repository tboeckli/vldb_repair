\section{Preliminaries}
\label{sec:preliminaries}

\paragraph{Relational model.}
In this paper, we assume the relational model~\cite{Codd1970}.
Within this context, let $\attrset$ be a countable set of attributes.
For each attribute $\attr \in \attrset$, let $\domain$ denote its \emph{domain} of possible values, excluding the \texttt{null}-value (or missing value).
Without loss of generality, we assume that each domain is infinite, yet countable (either ordered or unordered).
In the case of continuous domains, we enforce countability of $\domain$ by applying fixed-precision truncation on the domain values.
For instance, in Example~\ref{ex:runningexample}, each decimal number is truncated to two decimals places.
A \emph{(relation) schema} $\schema = \{\attrindex{1},\ldots,\attrindex{k}\}$ is a finite, non-empty subset of attributes.
A \emph{relation} $\relation$ over $\schema$ is a finite set $\relation \subseteq \domainindex{1} \times \ldots \times \domainindex{k}$, where $\domainindex{i}$ is the domain of $\attrindex{i}$.
Each element $\tuple \in \relation$ is called a \emph{tuple} $\tuple$ over $\schema$, and we assume that each $\schema$ includes a distinguished attribute \texttt{tid} whose values uniquely identify the tuples in $\relation$.
Finally, the \emph{projection} of a relation $\relation$ onto $X \subseteq \schema$ is denoted by $\projection{\relation}{X}$, or, analogously, by $\projection{\tuple}{X}$ for a single tuple $\tuple \in \relation$.
If $X$ is a singleton $\{\attr\}$, we denote the projection of $\relation$ over $\{\attr\}$ by $\projection{\relation}{\attr}$.

%\paragraph{Relational algebra.}
%Additionally, we consider \emph{(propositional) atoms} (or \emph{predicates}) $\atom$ over $\schema$ as Boolean conditions evaluated on tuples.
%Atoms can be combined into \emph{(propositional) formulas} $\formula$ by means of the Boolean operators $\wedge$ (and), $\vee$ (or) and $\neg$ (not).
%Given a relation $\relation$ and atom $\atom$ (resp.\ formula $\formula$), we write $\selection{\relation}{\atom}$ (resp.\ $\selection{\relation}{\formula}$) for the \emph{selection} (resp.\ \emph{generalized selection}) applied on $\relation$ with atom $\atom$ (resp.\ formula $\formula$).
%This operation returns the set of tuples in $\relation$ that \emph{satisfy} $\atom$ (resp. $\formula$), or $\selection{\relation}{\atom} = \{\tuple \mid \tuple \in \relation \wedge \tuple \models \atom \}$ (resp.\ $\selection{\relation}{\formula} = \{\tuple \mid \tuple \in \relation \wedge \tuple \models \formula \}$).

\paragraph{Selection rules.}
Within the relational model, a \emph{selection rule} is a type of integrity constraint that is used to enforce consistency at the level of tuples in a relation~\cite{Boeckling2022b,Boeckling2022a,Bronselaer2023}.
Formally, a selection rule $\srule$ over schema $\schema$ is defined by a (propositional) formula $\formula$ of the form
\begin{equation}
    \label{eq:sruleform}
    \neg (\atomindex{1} \wedge \ldots \wedge \atomindex{l}),
\end{equation}
in which each atom $\atom$ is either a \emph{constant} atom $\attr\;\operator\;\domainvalue$, a \emph{set} atom $\attr\;\operator\;\valueset$, or a \emph{variable} atom $\attrindex{i}\;\operator\;\attrindex{j}$.
Here, $\attr, \attrindex{i}, \attrindex{j} \in \schema$ share a common domain $\domain$, $\domainvalue \in \domain$ is a constant, and $\valueset \subseteq \domain$ is a finite set of constants.
For constant and variable atoms, the comparison operator $\operator$ is drawn from $\{\leq, <, \geq, >, =, \neq\}$ if $\domain$ is ordered, and from $\{=, \neq\}$ otherwise.
For set atoms, we have that $\operator \in \{\in, \notin\}$.
In the context of ordered domains, we consider an extension of variable atoms which allows us to explicitly express \emph{gaps} between the values of the attributes.
More specifically, these atoms take the form $\attrindex{i}\;\operator\; \successorfunction{n}{\attrindex{j}}$ in this case, with $n \in \mathbb{Z}$ and $\successor: \domain \rightarrow \domain$ the so-called \emph{successor function}.
This function returns for each $\domainvalue \in \domain$ the smallest value $\domainvalue' \in \domain$ such that $\domainvalue < \domainvalue'$ and there is no $\domainvalue'' \in \domain$ with $\domainvalue < \domainvalue'' < \domainvalue'$.
Moreover, we write $\successorpower{n}$ for the $n$-fold application of the successor function.
If $n < 0$, the application of the \emph{inverse} successor function, also referred to as the \emph{predecessor} function, is assumed.

A tuple $\tuple$ \emph{satisfies} a selection rule $\srule$ (written as $\tuple \models \srule$) if substitution of each attribute $\attr$ with its value $\projection{\tuple}{\attr}$ in the formula $\formula$ of $\srule$ causes $\formula$ to evaluate to true.
Otherwise, we say that $\tuple$ \emph{violates} or \emph{fails} $\srule$ (written as $\tuple \not\models \srule$).
Given a set of selection rules $\sruleset$, we denote the subset of rules violated by $\tuple$ as $\srulesetfailed{\sruleset}{\tuple} = \{\srule \mid \srule \in \sruleset \wedge \tuple \not\models \srule\}$.
A tuple $\tuple$ is \emph{consistent} with respect to $\sruleset$ if $\srulesetfailed{\sruleset}{\tuple} = \emptyset$ (written as $\tuple \models \sruleset$), and \emph{inconsistent} otherwise (written as $\tuple \not\models \sruleset$).
If an attribute $\attr \in \schema$ appears in one of the atoms of  $\srule$, $\attr$ is \emph{involved} in $\srule$, and the set of attributes that are involved in $\srule$ is denoted by $\involvedattr{\srule}$.

Finally, we distinguish three special types of selection rules.
First, a selection rule $\srule$ is a \emph{tautology} if its formula $\formula$ evaluates to always true, or $\formula \equiv \top$.
Second, $\srule$ is a \emph{contradiction} if its formula $\formula$ evaluates to always false, or $\formula \equiv \bot$.
Third, $\srule_r$ with formula $\formulaindex{r}$ is \emph{redundant} to $\srule_d$ with formula $\formulaindex{d}$ (or $\srule_d$ \emph{dominates} $\srule_r$) if $\formula_r \Rightarrow \formula_d$.

%\begin{itemize}
%    \item Variable and constant rules?
%    \item Univariate, multivariate rules?
%    \item Set cover, super cover
%   \item Equivalence?
%\end{itemize}

\paragraph{Cost functions and models.}
For an attribute $\attr \in \schema$ with domain $\domain$, a \emph{cost function} is a positive-definite function $\costfunctionattr{\attr} : \domain^2 \rightarrow \mathbb{N}$ that assigns, to each pair of values $(\domainvalue, \domainvalue') \in \domain^2$, the cost $\costfunctionattrvalues{\attr}{\domainvalue}{\domainvalue'}$ to change $\domainvalue$ into $\domainvalue'$ in a relation $\relation$ over $\schema$~\cite{Bronselaer2023}.
A cost function is called \emph{constant} if there exists a constant value $\alpha > 0$ such that, for all distinct pairs $(\domainvalue, \domainvalue') \in \domain^2$ with $\domainvalue \neq \domainvalue'$, we have $\costfunctionattrvalues{\attr}{\domainvalue}{\domainvalue'} = \alpha$.
It is often desirable, though not required, that the cost of changing the \texttt{null}-value into $\domainvalue \in \domain$ is constant, or, in other words, independent of $\domainvalue$.
Note, in this regard, that, while each attribute domain $\domain$ excludes the \texttt{null}-value,
we extend the domain of definition of each cost function to $(\domain \cup\{\texttt{null}\})^2$, thereby allowing updates involving missing values to be evaluated.
A \emph{cost model} $\costmodel$ for schema $\schema$ defines a cost function $\costfunctionattr{\attr}$ for each $\attr \in \schema$.
Given a cost model, the total cost of changing a tuple $\tuple$ into $\tuple'$ is then computed as
\begin{equation}
    \label{eq:costmodel}
    \costmodeltuples{\tuple}{\tuple'} = \sum_{\attr \in \schema} \costfunctionattrvalues{\attr}{\projection{\tuple}{\attr}}{\projection{\tuple'}{\attr}}.
\end{equation}